\documentclass[a4paper,12pt]{article}

%\usepackage{syntonly}
%\syntaxonly

\usepackage[utf8]{inputenc}
\usepackage[T1]{fontenc}
\usepackage{textcomp}

\usepackage{amsmath}
\usepackage{graphicx}

\title{Coursework for the Semantic Web module}
\author{Michal Grochmal
  $<$\href{mailto:grochmal@member.fsf.org}{grochmal@member.fsf.org}$>$
}
\date{\today}

\usepackage[pdftex,colorlinks=true]{hyperref}

\begin{document}
\maketitle

\section[Queation 1]{Answer to question 1}
\subsection[Describe]{Describe in English the contents of the document}
\begin{itemize}

\item[] \emph{Hamlet} is a \emph{drama}.

\item[] \emph{Sonet96} is a \emph{poem}.

\item[] \emph{"wrote"} is a poperty (role) which has domain \emph{writter} and
range \emph{literary\_content}.

\item[] \emph{poem} is a concept.

\item[] \emph{drama} is a concept.

\item[] \emph{Shakespear} wrote \emph{Hamlet} and wrote \emph{Sonet96}.
\emph{Shakespear} is a \emph{playwright}.

\end{itemize}

\newpage
\subsection[Draw]{Draw the graph representation of the doument}
\begin{figure}[!htp]
\centering
\includegraphics[width=\textwidth]{ex1_playwrights}
\caption{Graph representation of the RDF document}
\label{shakespear}
\end{figure}
Figure \ref{shakespear} shows the graph representation.

\subsection[RDF/S]{Represent the following in RDF/S}

\end{document}

