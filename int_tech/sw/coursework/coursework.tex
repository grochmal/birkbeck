\documentclass[a4paper,12pt]{article}

%\usepackage{syntonly}
%\syntaxonly

\usepackage[utf8]{inputenc}
\usepackage[T1]{fontenc}
\usepackage{textcomp}

\usepackage{amsmath}
\usepackage{graphicx}
\usepackage{fancyvrb}  % Verbatim instead of verbatim

\title{Coursework for the Semantic Web module}
\author{Michal Grochmal
  $<$\href{mailto:grochmal@member.fsf.org}{grochmal@member.fsf.org}$>$
}
\date{\today}

\usepackage[pdftex,colorlinks=true]{hyperref}

\begin{document}
\VerbatimFootnotes
\maketitle

\section[Question 1]{Answer to question 1}
\subsection[Describe]{Describe in English the contents of the document}
\begin{itemize}
\item[]\emph{Hamlet} is a \emph{drama}.

\item[]\emph{Sonet96} is a \emph{poem}.

\item[]\emph{"wrote"} is a property (role) which has domain \emph{writer} and
range \emph{literary\_content}.  Or in simpler words: Anything that
\emph{wrote} something is a \emph{writer} and that something it wrote is a
piece of \emph{literary\_content}.

\item[]\emph{poem} is a concept.

\item[]\emph{drama} is a concept.

\item[]\emph{Shakespear}, who is a \emph{Poet}, wrote \emph{Hamlet} and wrote
\emph{Sonet96}.  \emph{Shakespear} is a \emph{playwright} as well.
\end{itemize}

\subsection[Draw]{Draw the graph representation of the document}
\begin{figure}[!htp]
\centering
\includegraphics[width=\textwidth]{ex1_playwrights}
\caption{Graph representation of the RDF document}
\label{shakespear}
\end{figure}
Figure \ref{shakespear} shows the graph representation.

\subsection[RDF/S]{Represent the following in RDF/S}

"Poets and playwrights are writers." can easily be represented in RDF.  Using
turtle syntax we have:
\begin{Verbatim}[samepage=true]
:Poet        rdf:type  :Writer .
:Playwright  rdf:type  :Writer .
\end{Verbatim}

"Poets write poems; playwrights write dramas." can be represented in RDF.  We
can transliterate it to two separate statements:  "Poets write poems.
Playwrights write dramas", and write it down in turtle syntax:
\begin{Verbatim}[samepage=true]
:Poet        :write  :Poem .
:Playwright  :write  :Drama .
\end{Verbatim}

"Poets do not write RDF documents." \emph{Cannot} be represented in RDF/S as it
needs the complement (negation) operator.  i.e. in description logic it would
look like this: $Poet \sqsubseteq \forall write . \neg RDFDocument$.  And this
cannot be represented in RDF.

\section[Question 2]{Answer to question 2}

\subsection[Turtle]{Represent the information in turtle}
\begin{Verbatim}[samepage=true]
_b  rdf:type   :Person .
_b  :hasEmail  :bob@somewhere.uk .
_b  :playsGolfWith  _s .
_s  :isSpouseOf     _r .
_b  :isColleagueOf  _r .
_r  :hasEmail  :rob@elsewhere.uk .
_r  :hasEmail  :rob@somewhere.it .
\end{Verbatim}

\subsection[Graph]{Draw the graph representation}
Figure \ref{bobslife} shows the graph representation for this information.
\begin{figure}[!htp]
\centering
\includegraphics[width=\textwidth]{ex2_bob}
\caption{Representation of Bob's life}
\label{bobslife}
\end{figure}

\subsection[DL]{Represent the information using DL}
To represent this information in description logic (form a knowledge base)
logic we can use an empty \emph{Tbox} and the following \emph{ABox}:
\begin{align*}
                    b &: Person\\
(b, bob@somewhere.uk) &: hasEmail\\
(r, rob@elsewhere.uk) &: hasEmail\\
(r, rob@somewhere.it) &: hasEmail\\
               (b, r) &: isColleagueOf\\
               (b, s) &: playsGolfWith\\
               (s, r) &: isSpouseOf
\end{align*}

\section[Question 3]{Answer to question 3}

\subsection[Turtle]{Represent in RDF/S triples using turtle}
"Document 1 has been created by Paul on 29 November 2013." can be represented
as:
\begin{Verbatim}[samepage=true]
:Document1  :createdBy  :Paul .
:Document1  :createdOn  :2013-11-29 .
\end{Verbatim}

"Document 2 and document 3 have been created by the same (unknown) author." can
be represented as:
\begin{Verbatim}[samepage=true]
:Document2  :createdBy  _a .
:Document3  :createdBy  _a .
\end{Verbatim}

"Document 3 says that document 1 has been published by W3C." can be represented
as\footnote{We could add the \verb|:Document1  :publishedBy  :W3C| triple to
this representation.  But it is interpretation dependent, as some authors add
this triplet explicitly whilst others leave it implicit in the \emph{subject,
predicate and object} roles.}:
\begin{Verbatim}[samepage=true]
:Document3  :says  _s .
_s  :subject    :Document1 .
_s  :predicate  :publishedBy .
_s  :object     :W3C .
\end{Verbatim}

\subsection[SPARQL]{Represent in SPARQL}
We can represent the given query in the SPARQL syntax as:
\begin{Verbatim}[samepage=true]
SELECT ?a, ?d
WHERE { x?  :createdBy  ?a .
        x?  :createdOn  ?d .
      }
\end{Verbatim}

\section[Question 4]{Answer to question 4}

\subsection[RDF/S representation]{Represent in RDF/S if possible}
\begin{itemize}
\item[-]"John is a person." can be transliterated into "John is of type Person." and
written in RDF triples as:
\begin{Verbatim}[samepage=true]
:John  rdf:type  :Person .
\end{Verbatim}

\item[-]"Mary is John’s mother." can be transliterated into "Mary is the\\
mother of John." and written in RDF triples as:
\begin{Verbatim}[samepage=true]
:Mary  :motherOf  :John .
\end{Verbatim}

\item[-]"Persons are men and women." can be transliterated into "A Person is a man,
also a Person is a woman." which can be written in RDF/S triples
as\footnote{This is a very forceful way of transliterating that phrase, notably
because we say that every person is a Man and a Woman at the same time.
Another viable transliteration would be "A Person is either a Man or a Woman".
Yet, this transliteration cannot be represented using RDF triples as we need
the concept of disjoint classes, which is not present in RDF/S.}:
\begin{Verbatim}[samepage=true]
:Person  rdfs:subClassOf  :Man .
:Person  rdfs:subClassOf  :Woman .
\end{Verbatim}

\item[-]"Men and women are persons." can be transliterated into "Men are Persons and
Women are Persons." and written in RDF/S triplets as:
\begin{Verbatim}[samepage=true]
:Man    rdfs:subClassOf  :Person .
:Woman  rdfs:subClassOf  :Person .
\end{Verbatim}

\item[-]"All fathers are parents." can be transliterated into "Every father is parent
as well." and one form of writing that in RDF/S would be:
\begin{Verbatim}[samepage=true]
:fatherOf  rdfs:subPropertyOf  :parentOf .
\end{Verbatim}

\item[-]"Every mother is a woman." can be transliterated into "Something that is a
mother of something is a Woman." and written in RDF/S triples as:
\begin{Verbatim}[samepage=true]
:motherOf  rdfs:domain  :Woman .
\end{Verbatim}

\item[-]"Every person has a mother." cannot be represented in RDF/S.  To represent this
we need the concept of "exists" $(\exists)$ from Description Logic/OWL.

\item[-]"John’s mother is a lecturer." can be transliterated into "John has a mother;
John's mother is a Lecturer." and written in RDF triples as:
\begin{Verbatim}[samepage=true]
_m  :motherOf  :John .
_m  rdf:type   :Lecturer .
\end{Verbatim}

\item[-]"Every person has two parents." cannot be represented in RDF/S.  To represent
it we need the cardinality operators from extended DL and/or OWL 2.

\item[-]"Lecturers’ daughters are lecturers." cannot be represented in RDF/S.  To
represent this we need to intersect the class \emph{Lecturer} and the domain of
\emph{fatherOf}, but we cannot do intersection in RDF/S.

\item[-]"John and Sam have the same mother." can be transliterated into "Someone is
mother of John and this same someone is mother of Sam." and written in RDF
triples as:
\begin{Verbatim}[samepage=true]
_m  :motherOf  :John .
_m  :motherOf  :Sam .
\end{Verbatim}
\end{itemize}

\subsection[DL representation]{Represent as description logic}
We define our knowledge base as an \emph{ABox} and a \emph{Tbox}, first we
write the \emph{ABox}:
\begin{align*}
john         &: Person\\
mary         &: Lecturer\\
(mary, john) &: motherOf\\
(mary, sam)  &: motherOf
\end{align*}

And the \emph{TBox} looks like this:
\begin{align*}
               Person &\sqsubseteq Man \sqcup Woman\\
     Man \sqcap Woman &\sqsubseteq \emptyset\\
                  Man &\sqsubseteq Person\\
                Woman &\sqsubseteq Person\\
\exists fatherOf.\top &\sqsubseteq \exists parentOf.\top\\
\exists motherOf.\top &\sqsubseteq Woman\\
               Person &\sqsubseteq \exists motherOf^-.\top\\
Person &\sqsubseteq {}\leq2parentOf.\top {}\sqcap {}\geq2parentOf.\top\\
\exists parentOf^-.Lecturer \sqcap Woman &\sqsubseteq Lecturer
\end{align*}

\subsection[Dwarf]{Represent the statement in RDF/S using reification}
"the dwarf noticed that someone had eaten from its plate." can be represented,
using the turtle, syntax as:
\begin{Verbatim}[samepage=true]
:Dwarf  :noticed  _s .
_s  :subject    :Someone .
_s  :predicate  :hadEatenFrom .
_s  :object     :Plate .
:Plate  :belongsTo  :Dwarf .
\end{Verbatim}
The Figure \ref{dwarf} represents this same statement using a graph.
\begin{figure}[!htp]
\centering
\includegraphics[width=\textwidth]{ex4_dwarf}
\caption{Dwarf's perception}
\label{dwarf}
\end{figure}

\section[Question 5]{Answer to question 5}

\subsection[Publishing]{Ontology sketch for a publishing company}
The accompanying \emph{.owl} file contains the full ontology in a format
understood by proteg\'e.  In table \ref{sketch} we see a sketch of the
ontology.
\begin{table}[!htp]
\centering
\begin{tabular}{|l|l|l|l|}
\hline
self-standing            & modifiers         & properties   & definables \\
\hline \hline
Publisher                & Title             & hasTitle     & Author     \\
Person                   & ISBN              & hasISBN      & Editor     \\
- Woman                  & PubCategory       & isInCategory & Manager    \\
Publication              & - ScienceFiction  & hasAuthor    &\\
- Book                   & - ComputerScience & editedBy     &\\
- Journal                & PublicationForm   & printedIn    &\\
Location                 & - PhysicalPub     & hasPubForm   &\\
- Room                   & - DigitalPub      & published    &\\
- - room123 (individual) &                   & worksAt      &\\
                         &                   & authored     &\\
                         &                   & edited       &\\
\hline
\end{tabular}
\caption{Ontology sketch}
\label{sketch}
\end{table}

We can define the definables as:
\begin{description}
\item[Author] in description logic can be seen as: $ Author \equiv Person\\
\sqcap \exists authored.Publication $.  And in OWL functional syntax as:
\begin{Verbatim}[samepage=true]
EquivalentClasses(
    Author
    ObjectIntersectionOf(
        Person
        ObjectSomeValuesFrom(authored Publication)))
\end{Verbatim}

\item[Editor] in description logic can be seen as: $ Editor \equiv Person\\
\sqcap \exists edited.Publication \sqcap \exists worksAt.Publisher $.  And in
OWL functional syntax as:
\begin{Verbatim}[samepage=true]
EquivalentClasses(
    Editor
    ObjectIntersectionOf(
        Person
        ObjectSomeValuesFrom(edited Publication)
        ObjectSomeValuesFrom(worksAt Publisher)))
\end{Verbatim}

\item[Manager] in description logic can be seen as: $ Manager \equiv Person\\
\sqcap \exists worksAt.Publisher \sqcap \neg Editor $.  And in OWL functional
syntax as:
\begin{Verbatim}[samepage=true]
EquivalentClasses(
    Manager
    ObjectIntersectionOf(
        Person
        ObjectSomeValuesFrom(worksAt Publisher)
        ObjectComplementOf(Editor)))
\end{Verbatim}
\end{description}

The properties can be described as follows:
\begin{description}
\item[hasTitle]: domain Publication, range Title, functional.  A publication
have one title that distinguishes it from other publications, we are not
considering translations in this ontology.

\item[hasISBN]: domain Publication, range ISBN, functional.  An ISBN uniquely
identifies a publication.

\item[isInCategory]: domain Publication, range PubCategory.  The same
publication can belong to many categories at the same time.  Say a drama and
science fiction book.

\item[hasAuthor]: domain Publication, range Author, inverse of authored.  One
publication may be authored by more than one person.

\item[editedBy]: domain Publication, range Editor, inverse of edited.  One
publication can have several editors and the same editor will most probably
edit many publications.

\item[printedIn]: domain PhysicalPub, range Location, functional.  Only a
publication that was published in a physical form is printed (digital copies
are not printed), and every physical copy has been printed somewhere.

\item[hasPubForm]: domain Publication, range PubForm.  The same publication can
be published in many forms.  For example, a book can be published in physical
form and digital form.

\item[published]: domain Publisher, range Publication.  A publisher publishes
many publications.

\item[worksAt]: domain Person, range Publisher, functional.  Many people work
at the same publisher.  We are not considering that someone might do two jobs,
each in a different publisher, in the ontology.

\item[authored]: domain Author, range Publication, inverse of hasAuthor.  Same
author may write many publications.

\item[edited]: domain Editor, range Publication, inverse of editedBy.  Same
editor most likely edits many publications.
\end{description}

\subsection[Define classes]{Using the ontology define the classes}
"Multiple-author book." can be transliterated into "A Book that has more than
one Author." and is written in description logic as $ MultipleAuthorBook\\
\equiv Book \sqcap {} \geq 2hasAuthor.Author $.  In OWL functional syntax it
can be written as:
\begin{Verbatim}[samepage=true]
EquivalentClasses(
    MultiAuthorBook
    ObjectIntersectionOf(
        Book
        ObjectMinCardinality(2 hasAuthor)))
\end{Verbatim}

"Electronic publishing company for computer science." can be transliterated
into "A Publisher that publishes only electronic publications, and those
publications always fit in the computer science literary category.", this looks
in description logic as:
\begin{align*}
ElecPubCompSc \equiv Publisher \sqcap \exists &publishes.(Publication\\
                             &\sqcap \exists hasPubForm.DigitalPub\\
                             &\sqcap \forall hasPubForm.DigitalPub\\
                             &\sqcap \exists isInCategory.ComputerScience)\\
                               \sqcap \forall &publishes.(Publication\\
                             &\sqcap \exists hasPubForm.DigitalPub\\
                             &\sqcap \forall hasPubForm.DigitalPub\\
                             &\sqcap \exists isInCategory.ComputerScience)
\end{align*}
And in OWL functional syntax as:
\begin{Verbatim}[samepage=true]
EquivalentClasses(
    ElecPubCompSc
    ObjectIntersectionOf(
        Publisher
        ObjectSomeValuesFrom(
            publishes
            ObjectIntersectionOf(
                Publication
                ObjectSomeValuesFrom(hasPubForm DigitalPub)
                ObjectAllValuesFrom(hasPubForm DigitalPub)
                ObjectSomeValuesFrom(isInCategory ComputerScience)))
        ObjectAllValuesFrom(
            publishes
            ObjectIntersectionOf(
                Publication
                ObjectSomeValuesFrom(hasPubForm DigitalPub)
                ObjectAllValuesFrom(hasPubForm DigitalPub)
                ObjectSomeValuesFrom(isInCategory ComputerScience)))))
\end{Verbatim}

"Science fiction book printed in room 123 and edited by a woman." can be
transliterated into "A Book of category Science Fiction which has been printed
in the specific room, room 123.  And this book has been edited by a Woman."  In
description logic we can represent this as:
\begin{align*}
SfBookPr123EdFem \equiv Book &\sqcap \exists isInCategory.ScienceFiction\\
                             &\sqcap \exists printedIn.(value(room123))\\
                             &\sqcap \exists editedBy(Editor \sqcap Woman)
\end{align*}
And in OWL functional syntax as:
\begin{Verbatim}[samepage=true]
EquivalentClasses(
    SfBookPr123EdFem
    ObjectIntersectionOf(
        Book
        ObjectSomeValuesFrom(isInCategory ScienceFiction)
        ObjectSomeValuesFrom(
            printedIn
            ObectHasValue(room123))
        ObjectSomeValuesFrom(
            editedBy
            ObjectIntersectionOf(Editor Woman))))
\end{Verbatim}

\section[Question 6]{Answer to question 6}

\subsection[DL equivalents]{Write down description logic equivalents}
\begin{itemize}
\item[-]\begin{Verbatim}[samepage=true]
DisjointClasses(Animal Plant)
\end{Verbatim}
Is equivalent to: $ Animal \sqcap Plant \sqsubseteq \emptyset $

\item[-]\begin{Verbatim}[samepage=true]
ObjectPropertyDomain(eats Animal)
\end{Verbatim}
Is equivalent to: $ \exists eats.\top \sqsubseteq Animal $

\item[-]\begin{Verbatim}[samepage=true]
EquivalentClasses(Herbivore ObjectAllValuesFrom(eats Plant))
\end{Verbatim}
Is equivalent to: $ Herbivore \equiv \forall eats.Plant $

\item[-]\begin{Verbatim}[samepage=true]
EquivalentClasses(Carnivore ObjectAllValuesFrom(eats Animal))
\end{Verbatim}
Is equivalent to: $ Carnivore \equiv \forall eats.Animal $

\item[-]\begin{Verbatim}[samepage=true]
EquivalentClasses(CarnivorousPlant 
                  ObjectIntersectionOf(Plant Carnivore))
\end{Verbatim}
Is equivalent to: $ CarnivorousPlant \equiv Plant \sqcap Carnivore $
\end{itemize}

\end{document}

