\documentclass[a4paper,12pt]{article}

%\usepackage{syntonly}
%\syntaxonly

\usepackage[utf8]{inputenc}
\usepackage[T1]{fontenc}
\usepackage{textcomp}

\usepackage{amsmath}
\usepackage{amssymb}
\usepackage{graphicx}
\usepackage{fancyvrb}  % Verbatim instead of verbatim

\title{Coursework for the Semantic Web module}
\author{Michal Grochmal
  $<$\href{mailto:grochmal@member.fsf.org}{grochmal@member.fsf.org}$>$
}
\date{\today}

\usepackage[pdftex,colorlinks=true]{hyperref}

\begin{document}
\VerbatimFootnotes
\maketitle

\section[Question 1]{Answer to question 1}
\subsection[Describe]{Describe in English the contents of the document}
\begin{itemize}
\item[]\emph{Hamlet} is a \emph{drama}.

\item[]\emph{Sonet96} is a \emph{poem}.

\item[]\emph{"wrote"} is a property (role) which has domain \emph{writer} and
range \emph{literary\_content}.  Or in simpler words: Anything that
\emph{wrote} something is a \emph{writer} and that something it wrote is a
piece of \emph{literary\_content}.

\item[]\emph{poem} is a concept.

\item[]\emph{drama} is a concept.

\item[]\emph{Shakespear}, who is a \emph{Poet}, wrote \emph{Hamlet} and wrote
\emph{Sonet96}.  \emph{Shakespear} is a \emph{playwright} as well.
\end{itemize}

\subsection[Draw]{Draw the graph representation of the document}
\begin{figure}[!htp]
\centering
\includegraphics[width=\textwidth]{ex1_playwrights}
\caption{Graph representation of the RDF document}
\label{shakespear}
\end{figure}
Figure \ref{shakespear} shows the graph representation.

\subsection[RDF/S]{Represent the following in RDF/S}

"Poets and playwrights are writers." can easily be represented in RDF/S.
Using turtle syntax we have:
\begin{Verbatim}[samepage=true]
:Poet        rdfs:subClassOf  :Writer .
:Playwright  rdfs:subClassOf  :Writer .
\end{Verbatim}

"Poets write poems; playwrights write dramas." can be represented in RDF.  We
can transliterate it to two separate statements:  "Poets write poems.
Playwrights write dramas", and write it down in turtle syntax:
\begin{Verbatim}[samepage=true]
:Poet        :write  :Poem .
:Playwright  :write  :Drama .
\end{Verbatim}

"Poets do not write RDF documents." \emph{Cannot} be represented in RDF/S as it
needs the complement (negation) operator.  i.e. in description logic it would
look like this: $Poet \sqsubseteq \forall write . \neg RDFDocument$.  And this
cannot be represented in RDF.

\section[Question 2]{Answer to question 2}

\subsection[Turtle]{Represent the information in turtle}
\begin{Verbatim}[samepage=true]
_b  rdf:type   :Person .
_b  :hasEmail  :bob@somewhere.uk .
_b  :playsGolfWith  _s .
_s  :isSpouseOf     _r .
_b  :isColleagueOf  _r .
_r  :hasEmail  :rob@elsewhere.uk .
_r  :hasEmail  :rob@somewhere.it .
\end{Verbatim}

\subsection[Graph]{Draw the graph representation}
Figure \ref{bobslife} shows the graph representation for this information.
\begin{figure}[!htp]
\centering
\includegraphics[width=\textwidth]{ex2_bob}
\caption{Representation of Bob's life}
\label{bobslife}
\end{figure}

\subsection[DL]{Represent the information using DL}
To represent this information in description logic (form a knowledge base)
logic we can use an empty \emph{Tbox} and the following \emph{ABox}:
\begin{align*}
                    b &: Person\\
(b, bob@somewhere.uk) &: hasEmail\\
(r, rob@elsewhere.uk) &: hasEmail\\
(r, rob@somewhere.it) &: hasEmail\\
               (b, r) &: isColleagueOf\\
               (b, s) &: playsGolfWith\\
               (s, r) &: isSpouseOf
\end{align*}

\section[Question 3]{Answer to question 3}

\subsection[Turtle]{Represent in RDF/S triples using turtle}
"Document 1 has been created by Paul on 29 November 2013." can be represented
as:
\begin{Verbatim}[samepage=true]
:Document1  :createdBy  :Paul .
:Document1  :createdOn  :2013-11-29 .
\end{Verbatim}

"Document 2 and document 3 have been created by the same (unknown) author." can
be represented as:
\begin{Verbatim}[samepage=true]
:Document2  :createdBy  _a .
:Document3  :createdBy  _a .
\end{Verbatim}

"Document 3 says that document 1 has been published by W3C." can be represented
as\footnote{We could add the \verb|:Document1  :publishedBy  :W3C| triple to
this representation.  But it is interpretation dependent, as some authors add
this triplet explicitly whilst others leave it implicit in the \emph{subject,
predicate and object} roles.}:
\begin{Verbatim}[samepage=true]
:Document3  :says  _s .
_s  :subject    :Document1 .
_s  :predicate  :publishedBy .
_s  :object     :W3C .
\end{Verbatim}

\subsection[SPARQL]{Represent in SPARQL}
We can represent the given query in the SPARQL syntax as:
\begin{Verbatim}[samepage=true]
SELECT ?a, ?d
WHERE { x?  :createdBy  ?a .
        x?  :createdOn  ?d .
      }
\end{Verbatim}

\section[Question 4]{Answer to question 4}

\subsection[RDF/S representation]{Represent in RDF/S if possible}
\begin{itemize}
\item[-]"John is a person." can be transliterated into "John is of type Person." and
written in RDF triples as:
\begin{Verbatim}[samepage=true]
:John  rdf:type  :Person .
\end{Verbatim}

\item[-]"Mary is John’s mother." can be transliterated into "Mary is the\\
mother of John." and written in RDF triples as:
\begin{Verbatim}[samepage=true]
:Mary  :motherOf  :John .
\end{Verbatim}

\item[-]"Persons are men and women." can be transliterated into "A Person is a man,
also a Person is a woman." which can be written in RDF/S triples
as\footnote{This is a very forceful way of transliterating that phrase, notably
because we say that every person is a Man and a Woman at the same time.
Another viable transliteration would be "A Person is either a Man or a Woman".
Yet, this transliteration cannot be represented using RDF triples as we need
the concept of disjoint classes, which is not present in RDF/S.}:
\begin{Verbatim}[samepage=true]
:Person  rdfs:subClassOf  :Man .
:Person  rdfs:subClassOf  :Woman .
\end{Verbatim}

\item[-]"Men and women are persons." can be transliterated into "Men are Persons and
Women are Persons." and written in RDF/S triplets as:
\begin{Verbatim}[samepage=true]
:Man    rdfs:subClassOf  :Person .
:Woman  rdfs:subClassOf  :Person .
\end{Verbatim}

\item[-]"All fathers are parents." can be transliterated into "Every father is parent
as well." and one form of writing that in RDF/S would be:
\begin{Verbatim}[samepage=true]
:fatherOf  rdfs:subPropertyOf  :parentOf .
\end{Verbatim}

\item[-]"Every mother is a woman." can be transliterated into "Something that is a
mother of something is a Woman." and written in RDF/S triples as:
\begin{Verbatim}[samepage=true]
:motherOf  rdfs:domain  :Woman .
\end{Verbatim}

\item[-]"Every person has a mother." cannot be represented in RDF/S.  To represent this
we need the concept of "exists" $(\exists)$ from Description Logic/OWL.

\item[-]"John’s mother is a lecturer." can be transliterated into "John has a mother;
John's mother is a Lecturer." and written in RDF triples as:
\begin{Verbatim}[samepage=true]
_m  :motherOf  :John .
_m  rdf:type   :Lecturer .
\end{Verbatim}

\item[-]"Every person has two parents." cannot be represented in RDF/S.  To represent
it we need the cardinality operators from extended DL and/or OWL 2.

\item[-]"Lecturers’ daughters are lecturers." cannot be represented in RDF/S.  To
represent this we need to intersect the class \emph{Lecturer} and the domain of
\emph{fatherOf}, but we cannot do intersection in RDF/S.

\item[-]"John and Sam have the same mother." can be transliterated into "Someone is
mother of John and this same someone is mother of Sam." and written in RDF
triples as:
\begin{Verbatim}[samepage=true]
_m  :motherOf  :John .
_m  :motherOf  :Sam .
\end{Verbatim}
\end{itemize}

\subsection[DL representation]{Represent as description logic}
We define our knowledge base as an \emph{ABox} and a \emph{Tbox}, first we
write the \emph{ABox}:
\begin{align*}
john         &: Person\\
mary         &: Lecturer\\
(mary, john) &: motherOf\\
(mary, sam)  &: motherOf
\end{align*}

And the \emph{TBox} looks like this:
\begin{align*}
               Person &\sqsubseteq Man \sqcup Woman\\
     Man \sqcap Woman &\sqsubseteq \emptyset\\
                  Man &\sqsubseteq Person\\
                Woman &\sqsubseteq Person\\
\exists fatherOf.\top &\sqsubseteq \exists parentOf.\top\\
\exists motherOf.\top &\sqsubseteq Woman\\
               Person &\sqsubseteq \exists motherOf^-.\top\\
Person &\sqsubseteq {}\leq2parentOf.\top {}\sqcap {}\geq2parentOf.\top\\
\exists parentOf^-.Lecturer \sqcap Woman &\sqsubseteq Lecturer
\end{align*}

\subsection[Dwarf]{Represent the statement in RDF/S using reification}
"the dwarf noticed that someone had eaten from its plate." can be represented,
using the turtle, syntax as:
\begin{Verbatim}[samepage=true]
:Dwarf  :noticed  _s .
_s  :subject    :Someone .
_s  :predicate  :hadEatenFrom .
_s  :object     :Plate .
:Plate  :belongsTo  :Dwarf .
\end{Verbatim}
The Figure \ref{dwarf} represents this same statement using a graph.
\begin{figure}[!htp]
\centering
\includegraphics[width=\textwidth]{ex4_dwarf}
\caption{Dwarf's perception}
\label{dwarf}
\end{figure}

\section[Question 5]{Answer to question 5}

\subsection[Publishing]{Ontology sketch for a publishing company}
The accompanying \emph{.owl} file contains the full ontology in a format
understood by proteg\'e.  In table \ref{sketch} we see a sketch of the
ontology.  In this sketch ontology \emph{room123} is an individual.
\begin{table}[!htp]
\centering
\begin{tabular}{|l|l|l|l|}
\hline
self-standing        & modifiers         & properties   & definables  \\
\hline \hline
Publisher            & Title             & hasTitle     & Author      \\
Person               & ISBN              & hasISBN      & Editor      \\
- Woman              & PubCategory       & isInCategory & Manager     \\
Publication          & - ScienceFiction  & hasAuthor    & PhysicalPub \\
- Book               & - ComputerScience & editedBy     & DigitalPub  \\
- Journal            & PublicationForm   & printedIn    &\\
Location             & - Physical        & hasPubForm   &\\
- Room               & - Digital         & published    &\\
- - room123 (indiv.) &                   & worksAt      &\\
                     &                   & authored     &\\
                     &                   & edited       &\\
\hline
\end{tabular}
\caption{Ontology sketch}
\label{sketch}
\end{table}

We can define the definables as:
\begin{description}
\item[Author] in description logic can be seen as: $ Author \equiv Person\\
\sqcap \exists authored.Publication $.  And in OWL functional syntax as:
\begin{Verbatim}[samepage=true]
EquivalentClasses(
    Author
    ObjectIntersectionOf(
        Person
        ObjectSomeValuesFrom(authored Publication)))
\end{Verbatim}

\item[Editor] in description logic can be seen as: $ Editor \equiv Person\\
\sqcap \exists edited.Publication \sqcap \exists worksAt.Publisher $.  And in
OWL functional syntax as:
\begin{Verbatim}[samepage=true]
EquivalentClasses(
    Editor
    ObjectIntersectionOf(
        Person
        ObjectSomeValuesFrom(edited Publication)
        ObjectSomeValuesFrom(worksAt Publisher)))
\end{Verbatim}

\item[Manager] in description logic can be seen as: $ Manager \equiv Person\\
\sqcap \exists worksAt.Publisher \sqcap \neg Editor $.  And in OWL functional
syntax as:
\begin{Verbatim}[samepage=true]
EquivalentClasses(
    Manager
    ObjectIntersectionOf(
        Person
        ObjectSomeValuesFrom(worksAt Publisher)
        ObjectComplementOf(Editor)))
\end{Verbatim}

\item[PhysicalPub] in description logic can be seen as: $ PhysicalPub \equiv\\
Publication \sqcap \exists hasPubForm Physical $.  And in OWL functional syntax
as:
\begin{Verbatim}[samepage=true]
EquivalentClasses(
    PhysicalPub
    ObjectIntersectionOf(
        Publication
        ObjectSomeValuesFrom(hasPubForm Physical)))
\end{Verbatim}

\item[DigitalPub] in description logic can be seen as: $ DigitalPub \equiv\\
Publication \sqcap \exists hasPubForm Digital $.  And in OWL functional syntax
as:
\begin{Verbatim}[samepage=true]
EquivalentClasses(
    DigitalPub
    ObjectIntersectionOf(
        Publication
        ObjectSomeValuesFrom(hasPubForm Digital)))
\end{Verbatim}
\end{description}

The properties can be described as follows:
\begin{description}
\item[hasTitle]: domain Publication, range Title, functional.  A publication
have one title that distinguishes it from other publications, we are not
considering translations in this ontology.

\item[hasISBN]: domain Publication, range ISBN, functional.  An ISBN uniquely
identifies a publication.

\item[isInCategory]: domain Publication, range PubCategory.  The same
publication can belong to many categories at the same time.  Say a drama and
science fiction book.

\item[hasAuthor]: domain Publication, range Author, inverse of authored.  One
publication may be authored by more than one person.

\item[editedBy]: domain Publication, range Editor, inverse of edited.  One
publication can have several editors and the same editor will most probably
edit many publications.

\item[printedIn]: domain PhysicalPub, range Location, functional.  Only a
publication that was published in a physical form is printed (digital copies
are not printed), and every physical copy has been printed somewhere.

\item[hasPubForm]: domain Publication, range PublicationForm.  The same
publication can be published in many forms.  For example, a book can be
published in physical form and digital form.

\item[published]: domain Publisher, range Publication, inverse functional.  A
publisher publishes many publications, yet when the same material is published
by another publisher it becomes a different publication.

\item[worksAt]: domain Person, range Publisher, functional.  Many people work
at the same publisher.  We are not considering that someone might do two jobs,
each in a different publisher, in the ontology.

\item[authored]: domain Author, range Publication, inverse of hasAuthor.  Same
author may write many publications.

\item[edited]: domain Editor, range Publication, inverse of editedBy.  Same
editor most likely edits many publications.
\end{description}

\subsection[Define classes]{Using the ontology define the classes}
"Multiple-author book." can be transliterated into "A Book that has more than
one Author." and is written in description logic as $ MultipleAuthorBook\\
\equiv Book \sqcap {} \geq 2hasAuthor.Author $.  In OWL functional syntax it
can be written as:
\begin{Verbatim}[samepage=true]
EquivalentClasses(
    MultiAuthorBook
    ObjectIntersectionOf(
        Book
        ObjectMinCardinality(2 hasAuthor)))
\end{Verbatim}

"Electronic publishing company for computer science." can be transliterated
into "A Publisher that publishes only electronic publications, and those
publications always fit in the computer science literary category.", this looks
in description logic as:
\begin{align*}
ElecPubCompSc \equiv Publisher \sqcap \exists &publishes.(Publication\\
                             &\sqcap \exists hasPubForm.DigitalPub\\
                             &\sqcap \forall hasPubForm.DigitalPub\\
                             &\sqcap \exists isInCategory.ComputerScience)\\
                               \sqcap \forall &publishes.(Publication\\
                             &\sqcap \exists hasPubForm.DigitalPub\\
                             &\sqcap \forall hasPubForm.DigitalPub\\
                             &\sqcap \exists isInCategory.ComputerScience)
\end{align*}
And in OWL functional syntax as:
\begin{Verbatim}[samepage=true]
EquivalentClasses(
    ElecPubCompSc
    ObjectIntersectionOf(
        Publisher
        ObjectSomeValuesFrom(
            publishes
            ObjectIntersectionOf(
                Publication
                ObjectSomeValuesFrom(hasPubForm DigitalPub)
                ObjectAllValuesFrom(hasPubForm DigitalPub)
                ObjectSomeValuesFrom(isInCategory ComputerScience)))
        ObjectAllValuesFrom(
            publishes
            ObjectIntersectionOf(
                Publication
                ObjectSomeValuesFrom(hasPubForm DigitalPub)
                ObjectAllValuesFrom(hasPubForm DigitalPub)
                ObjectSomeValuesFrom(isInCategory ComputerScience)))))
\end{Verbatim}

"Science fiction book printed in room 123 and edited by a woman." can be
transliterated into "A Book of category Science Fiction which has been printed
in the specific room, room 123.  And this book has been edited by a Woman."  In
description logic we can represent this as:
\begin{align*}
SfBookPr123EdFem \equiv Book &\sqcap \exists isInCategory.ScienceFiction\\
                             &\sqcap \exists printedIn.(value(room123))\\
                             &\sqcap \exists editedBy(Editor \sqcap Woman)
\end{align*}
And in OWL functional syntax as:
\begin{Verbatim}[samepage=true]
EquivalentClasses(
    SfBookPr123EdFem
    ObjectIntersectionOf(
        Book
        ObjectSomeValuesFrom(isInCategory ScienceFiction)
        ObjectSomeValuesFrom(
            printedIn
            ObectHasValue(room123))
        ObjectSomeValuesFrom(
            editedBy
            ObjectIntersectionOf(Editor Woman))))
\end{Verbatim}

\section[Question 6]{Answer to question 6}

\subsection[DL equivalents]{Write down description logic equivalents}
\begin{itemize}
\item[-]\begin{Verbatim}[samepage=true]
DisjointClasses(Animal Plant)
\end{Verbatim}
Is equivalent to: $ Animal \sqcap Plant \sqsubseteq \bot $

\item[-]\begin{Verbatim}[samepage=true]
ObjectPropertyDomain(eats Animal)
\end{Verbatim}
Is equivalent to: $ \exists eats.\top \sqsubseteq Animal $

\item[-]\begin{Verbatim}[samepage=true]
EquivalentClasses(Herbivore ObjectAllValuesFrom(eats Plant))
\end{Verbatim}
Is equivalent to: $ Herbivore \equiv \forall eats.Plant $

\item[-]\begin{Verbatim}[samepage=true]
EquivalentClasses(Carnivore ObjectAllValuesFrom(eats Animal))
\end{Verbatim}
Is equivalent to: $ Carnivore \equiv \forall eats.Animal $

\item[-]\begin{Verbatim}[samepage=true]
EquivalentClasses(CarnivorousPlant 
                  ObjectIntersectionOf(Plant Carnivore))
\end{Verbatim}
Is equivalent to: $ CarnivorousPlant \equiv Plant \sqcap Carnivore $
\end{itemize}

\subsection[Hierarchy]{Hierarchy computed by proteg\'e}
The table \ref{protege} shows the hierarchy after the original ontology is
computed in proteg\'e and the hierarchy after the fixes done in the following
sections.
\begin{table}[!htp]
\centering
\begin{tabular}{|l|l|l|l|}
\hline
original ontology             &
after section \ref{fixforall} &
after section \ref{fixdomain} \\
\hline \hline
Thing                & Thing              & Thing                \\
- Animal             & - Animal           & - Animal             \\
- Carnivore          & - Carnivore        & - Carnivore          \\
- - CarnivorousPlant & - Herbivore        & - - CarnivorousPlant \\
- - Plant            & - Plant            & - Herbivore          \\
- Herbivore          & Nothing            & - Plant              \\
- - CarnivorousPlant & - CarnivorousPlant & - - CarnivorousPlant \\
- - Plant            &                    &\\
\hline
\end{tabular}
\caption{Natural taxonomy ontology}
\label{protege}
\end{table}

\subsection[Explain misbehaviour]{Explain plants' definition in the ontology}
A \emph{Plant} do not eat anything in this model.  As the definition of
\emph{Carnivore} $ (\forall eats.Animal) $ do not specify that the carnivore
must actually eat something, a plant is a perfectly good example of a
carnivore.  The analogous reasoning explains why a plant is a \emph{Herbivore},
as the definition of herbivore do not specify that a member of it must eat
something.

\subsection[First fix]{How the problem can be fixed}
\label{fixforall}
The definition of both Carnivore and Herbivore shall specify that members of
these classes must actually eat something.  The new definitions look like this:
\begin{align*}
Carnivore &\equiv \forall eats.Animal \sqcap \exists eats.Animal\\
Herbivore &\equiv \forall eats.Plant  \sqcap \exists eats.Plant
\end{align*}

\subsection[Fixed ontology]{Another problem with the ontology}
\label{fixdomain}
Another problem with the ontology is that carnivorous plants cannot be defined.
Carnivorous plants are plants that are carnivorous, i.e. that eat animals, but
the definition of the domain of \emph{eats} do not allow plants to eat.  To fix
this the restriction on the domain of \emph{eats} is removed from the ontology.
The full ontology, after all fixes, can be found in accompanying \emph{.owl}
file.

\section[Question 7]{Answer to question 7}

Consider the TBox $T$:
\begin{align*}
                 \neg A &\sqsubseteq \forall R.\neg B\\
\exists R.(\exists R.C) &\sqsubseteq A \sqcup B
\end{align*}
And the interpretation $ I = (\Delta^I, \cdot^I) $:
\begin{align*}
\Delta^I &= \{a,b,c,d,e,f\}\\
     A^I &= \{a,b,f\}\\
     B^I &= \{b,d\}\\
     C^I &= \{e\}\\
     R^I &= \{(a,b),(a,c),(d,c),(c,e)\}
\end{align*}

\subsection[Model]{Is $I$ a model of $T$?}
\label{itproof}
For $I$ to be a model of $T$ the interpretation must satisfy both axioms in the
TBox.  As for the first axiom:
\begin{align*}
        \neg A &\sqsubseteq \forall R.\neg B\\
\neg \{a,b,f\} &\sqsubseteq \forall R.\neg \{b,f\}\\
     \{c,d,e\} &\sqsubseteq \forall R.\{a,c,d,e\}\\
     \{c,d,e\} &\sqsubseteq \{b,c,d,e,f\}
\end{align*}
Which is true.  The second axiom:
\begin{align*}
    \exists R.(\exists R.C) &\sqsubseteq A \sqcup B\\
\exists R.(\exists R.\{e\}) &\sqsubseteq \{a,b,f\} \sqcup \{b,d\}\\
            \exists R.\{c\} &\sqsubseteq \{a,b,d,f\}\\
                    \{a,d\} &\sqsubseteq \{a,b,d,f\}
\end{align*}
Which is true as well.  As both axioms of TBox $T$ hold in the interpretation
$I$ we can say that $I$ is a model of $T$.

\subsection[Satisfiability]{Is $\exists R.(\exists R.A)$ unsatisfiable in $T$?}
We first check whether the concept $ \exists R.(\exists R.A) $ is empty in $I$:
\begin{align*}
&\exists R.(\exists R.A)\\
&\exists R.(\exists R.\{a,b,f\})\\
&\exists R.(\{a\})\\
&\emptyset
\end{align*}
Therefore $ \exists R.(\exists R.A) $ is empty in $I$.  Yet, for this concept
to be unsatisfiable with respect to $T$ it must be unsatisfiable for all
interpretations that model $T$.  If we can find an interpretation that models
$T$ and in which $ \exists R.(\exists R.A) $ is not empty then this concept is
satisfiable in $T$.  One such interpretation is:
\begin{equation*}
I' = I \cup \{ R = \{(a,a)\} \}
\end{equation*}
i.e. the same interpretation as $I$ but with an extra relation $aRa$.  The
proof that this interpretation is a model of $T$ is exactly the same as the
proof for the interpretation $I$ in section \ref{itproof}.  In the
interpretation $I'$ the concept $ \exists R.(\exists R.A) $ is:
\begin{align*}
&\exists R.(\exists R.A)\\
&\exists R.(\exists R.\{a,b,f\})\\
&\exists R.(\{a\})\\
&\{a\}
\end{align*}
Therefore there exists at least one interpretation that models $T$ and
satisfies $ \exists R.(\exists R.A) $.  So it is not true that this concept is
unsatisfiable with relation to $T$.

\subsection[Consequence]{Is $ T \models C \sqsubseteq \forall R.A $?}
We first check whether the axiom $ C \sqsubseteq \forall R.A $ is satisfied in
$I$:
\begin{align*}
    C &\sqsubseteq \forall R.A\\
\{e\} &\sqsubseteq \{b,e,f\}
\end{align*}
Therefore $ C \sqsubseteq \forall R.A $ is satisfiable in the interpretation
$I$.  Yet, for this axiom to be a logical consequence of $T$ it needs to be
satisfiable for all interpretations that model $T$.  If we have an
interpretation that models $T$ but do not satisfy this axiom then it is not a
logical consequence of $T$.  One such interpretation is:
\begin{equation*}
I'' = I \cup \{ C = \{d\} \}
\end{equation*}
i.e. the same relation as $I$ but where the individual $b$ belongs to the class
$C$ (apart from other classes it might already belong).  The proof that this
interpretation is a model of $T$ is exactly the same as the proof for the
interpretation $I$ in section \ref{itproof}.  In the interpretation $I''$ the
axiom $ C \sqsubseteq \forall R.A $ is:
\begin{align*}
      C &\sqsubseteq \forall R.A\\
\{d,e\} &\sqsubseteq \{b,e,f\}
\end{align*}
Which is not true.  Therefore there exists at least on interpretation that
models $T$ but do not satisfy the axiom $ C \sqsubseteq \forall R.A $.  So it
is not true that $ T \models C \sqsubseteq \forall R.A $.

\section[Question 8]{Answer to question 8}

Consider the ABox $A$
\begin{align*}
(ron,claudia)   &: likes\\
(ron,peter)     &: likes\\
(claudia,peter) &: isNeighbourOf\\
(peter,andrea)  &: isNeighbourOf\\
claudia         &: Blond\\
andrea          &: \neg Blond
\end{align*}

\subsection[Model]{Does $A$ have a model?}
\label{roninterpretation}
Yes we can build an interpretation that is a model of $A$, for example the
interpretation $ I = (\Delta^I, \cdot^I) $:
\begin{align*}
       \Delta^I &= \{ron, claudia, peter, andrea\}\\
        Blond^I &= \{claudia\}\\
        likes^I &= \{(ron,claudia),(ron,peter)\}\\
isNeighbourOf^I &= \{(claudia,peter),(peter,andrea)\}
\end{align*}

\subsection[Exists]{Is $ron$ an instance of the concept}
\begin{equation*}
\exists likes.(Blond \sqcap \exists isNeighbourOf. \neg Blond)
\end{equation*}
To check that $ron$ is an instance of this concept we need to see whether this
is true for $ron$ in all interpretations that model the ABox $A$.  As in all
interpretations that model the ABox $A$ will have the relations:
\begin{equation*}
likes = \{(ron,claudia),(ron,peter)\}
\end{equation*}
Where we know that $claudia$ is $Blond$, but we do not know whether $peter$ is
$Blond$ or $\neg Blond$; we shall check if the concept is true for $peter$
being $Blond$ and for $peter$ being $\neg Blond$:
\begin{description}
\item[peter is Blond:] In this case the concept is true because $ron$ likes
$peter$.  $peter$ is $Blond$ and is $isNeighbourOf$ $andrea$, who is $\neg
Blond$.

\item[peter is not Blond:] In this case the concept is true because $ron$ likes
$claudia$.  $claudia$ is $Blond$ and $isNeighbourOf$ $peter$, who is $\neg
Blond$.
\end{description}

As in both cases the concept is true and that any interpretation that models
$A$ must have one of these cases we can say that: it is true that $ron$ is an
instance of the concept:
\begin{equation*}
\exists likes.(Blond \sqcap \exists isNeighbourOf. \neg Blond)
\end{equation*}

\subsection[For all]{Is ron an instance of the concept}
\begin{equation*}
\exists likes.(\exists isNeighbourOf.(\forall isNeighbourOf.Blond))
\end{equation*}
Once again, to check that $ron$ is an instance of this concept we need to see
whether this is true for $ron$ in all interpretations that model the ABox $A$.

Although this is true in the interpretation $I$ (defined in section
\ref{roninterpretation}):  Where $ron$ $likes$ $peter$; $peter$ $isNeighbourOf$
$andrea$; and $andrea$ has no neighbours, therefore all her/his neighbours are
$Blond$.  The ABox $A$ (and description logic in general) uses an open world
assumption, therefore any interpretation where $andrea$ has neighbours that are
$\neg Blond$ still models $A$.

As we can construct an interpretation that models the ABox $A$ and in this
interpretation $andrea$ has a neighbour that is $\neg Blond$, in such
interpretation the given concept is not true for $ron$.  From this we can say
that $ron$ is not an instance of the concept:
\begin{equation*}
\exists likes.(\exists isNeighbourOf.(\forall isNeighbourOf.Blond))
\end{equation*}

\section[Question 9]{Answer to question 9}

Consider the concepts:
\begin{align*}
C &\equiv \neg A \sqcap \exists R.A \sqcap \neg(\forall R.(A \sqcup \neg B))\\
D &\equiv B \sqcap \exists R.(A \sqcap B) \sqcap \neg(\exists R.A)
\end{align*}

\subsection[NNF]{Transform the concepts into negation normal form}
Transforming concept $C$:
\begin{align*}
\neg A \sqcap \exists R.A &\sqcap \neg(\forall R.(A \sqcup \neg B))\\
\neg A \sqcap \exists R.A &\sqcap \exists R.\neg(A \sqcup \neg B)\\
\neg A \sqcap \exists R.A &\sqcap \exists R.(\neg A \sqcap \neg \neg B)\\
\neg A \sqcap \exists R.A &\sqcap \exists R.(\neg A \sqcap B)
\end{align*}

Transforming concept $D$:
\begin{align*}
B \sqcap \exists R.(A \sqcap B) &\sqcap \neg(\exists R.A)\\
B \sqcap \exists R.(A \sqcap B) &\sqcap \forall R.\neg A
\end{align*}

\subsection[Concept C]{Is concept $C$ satisfiable?}
\label{interpretationforc}
A concept is satisfiable if we can construct an interpretation for it, let's
try to construct an interpretation for concept $C$.  Lets call this
interpretation $S$:
\begin{align*}
S_0 &= \{x : \neg A \sqcap \exists R.A \sqcap \exists  R.(\neg A \sqcap B)\}\\
S_0 \to_\sqcap S_1 &= S_0 \cup \{ x : \neg A, x : \exists R.A
                                , x : \exists R.(\neg A \sqcap B) \}\\
S_1 \to_\exists S_2 &= S_1 \cup \{ x : \neg A, (x,y) : R, y : A
                                 , x : \exists R.(\neg A \sqcap B) \}\\
S_2 \to_\exists S_3 &= S_2 \cup \{ x : \neg A, (x,y) : R, y : A
                                 , (x,z) : R, z : \neg A \sqcap B \}\\
S_3 \to_\sqcap S_4 &= S_3 \cup \{ x : \neg A, (x,y) : R, y : A
                                , (x,z) : R, z : \neg A, z : B \}
\end{align*}

From this we can write the interpretation $S$ as $ S = (\Delta^S, \cdot^S) $
where:
\begin{align*}
\Delta^S &= \{x,y,z\}\\
     A^S &= \{y\}\\
     B^S &= \{z\}\\
     R^S &= \{(x,y),(x,z)\}
\end{align*}

Since interpretation $S$ satisfies the concept $C$, concept $C$ is satisfiable.

\subsection[Concept D]{Is concept $D$ satisfiable?}
A concept is satisfiable if we can construct an interpretation for it, let's
try to construct an interpretation for concept $D$.  Lets call this
interpretation $I$:
\begin{align*}
I_0 &= \{x : B \sqcap \exists R.(A \sqcap B) \sqcap \forall R.\neg A\}\\
I_0 \to_\sqcap I_1 &= I_0 \cup \{ x : B, x : \exists R.(A \sqcap B)
                                , x : \forall R.\neg A\}\\
I_1 \to_\exists I_2 &= I_1 \cup \{ x : B, (x,y) : R
                                 , y : A \sqcap B, x : \forall R.\neg A\}\\
I_2 \to_\sqcap I_3 &= I_2 \cup \{ x : B, (x,y) : R
                                , y : A, y : B, x : \forall R.\neg A\}\\
I_3 \to_\forall I_4 &= I_3 \cup \{ x : B, (x,y) : R
                                 , y : A, y : B, y : \neg A\} &(contradiction!)
\end{align*}

The tableau method falls in a contradiction when trying to build an
interpretation for concept $D$ (as both $ y : A $ and $ y : \neg A $ must be
present in an interpretation).  Since we cannot build an interpretation for
concept $D$ we can say that concept $D$ is unsatisfiable.

\subsection[Subsumption]{Check if subsumption $C \sqsubseteq D$ holds}
For the the subsumption $ C \sqsubseteq D $ to be true it must hold that for
all interpretations $I$: $ C^I \subseteq D^I $ is true.  There is no limitation
on the form of the interpretations $I$ as we are not checking the subsumption
against a TBox (although we could say we are checking the subsumption w.r.t an
empty TBox).

If it is true that $ C^I \subseteq D^I \mid \forall I$ then the concept $ C
\sqcap \neg D $ must be unsatisfiable (if set $C$ is within set $D$ then the
intersection of $C$ with the complement of $D$ must be the empty set).
Therefore we shall check the satisfiability of $ C \sqcap \neg D $, first we
need its negation normal form:
\begin{align*}
        \neg A \sqcap \exists R.A \sqcap \exists R.(\neg A \sqcap B)
&\sqcap \neg (B \sqcap \exists R.(A \sqcap B) \sqcap \forall R.\neg A)\\
        \neg A \sqcap \exists R.A \sqcap \exists R.(\neg A \sqcap B)
&\sqcap \neg (B \sqcap \exists R.(A \sqcap B)) \sqcup \neg \forall R.\neg A\\
        \neg A \sqcap \exists R.A \sqcap \exists R.(\neg A \sqcap B)
&\sqcap \neg B \sqcup \neg \exists R.(A \sqcap B) \sqcup \exists R.A\\
        \neg A \sqcap \exists R.A \sqcap \exists R.(\neg A \sqcap B)
&\sqcap \neg B \sqcup \forall R.\neg (A \sqcap B) \sqcup \exists R.A\\
        \neg A \sqcap \exists R.A \sqcap \exists R.(\neg A \sqcap B)
&\sqcap \neg B \sqcup \forall R.(\neg A \sqcup \neg B) \sqcup \exists R.A\\
\end{align*}

Lets try to construct an interpretation that satisfies the concept above, lets
call this interpretation $P$:
\begin{eqnarray*}
P_0 &=& \{ x : \neg A \sqcap \exists R.A \sqcap \exists R.(\neg A \sqcap B)\\
    & &  \sqcap \neg B \sqcup
         \forall R.(\neg A \sqcup \neg B) \sqcup \exists R.A\}\\
P_0 \to_\sqcap P_1 &= P_0 \cup& \{ x : \neg A, x : \exists R.A\\
                              & &, x : \exists R.(\neg A \sqcap B)
                                 , x : \neg B \sqcup
                                       \forall R.(\neg A \sqcup \neg B)
                                       \sqcup \exists R.A\}\\
P_1 \to_\exists P_2 &= P_1 \cup& \{ x : \neg A, x : (x,y) : R, y : A\\
                               & &, x : \exists R.(\neg A \sqcap B)
                                  , x : \neg B \sqcup
                                        \forall R.(\neg A \sqcup \neg B)
                                        \sqcup \exists R.A\}\\
P_2 \to_\exists P_3 &= P_2 \cup& \{ x : \neg A
                                  , x : (x,y) : R, y : A, (x,z) : R\\
                               & &, z : \neg A \sqcap B
                                  , x : \neg B \sqcup
                                        \forall R.(\neg A \sqcup \neg B)
                                        \sqcup \exists R.A\}\\
P_3 \to_\sqcap P_4 &= P_3 \cup& \{ x : \neg A
                                 , x : (x,y) : R, y : A , (x,z) : R\\
                              & &, z : \neg A, z : B
                                 , x : \neg B \sqcup
                                       \forall R.(\neg A \sqcup \neg B)
                                       \sqcup \exists R.A\}\\
P_4 \to_\sqcap P_{5,1} &= P_4 \cup& \{ x : \neg A , x : (x,y) : R\\
                                  & &, y : A , (x,z) : R
                                     , z : \neg A, z : B, x : \neg B \}
\end{eqnarray*}
And we can stop in $ P_{5,1} $ because we found an interpretation that
satisfies $ C \sqsubseteq D $.  This interpretation is $ P = (\Delta^P,
\cdot^P) $:
\begin{align*}
\Delta^P &= \{x,y,z\}\\
     A^P &= \{y\}\\
     B^P &= \{z\}\\
     R^P &= \{(x,y),(x,z)\}
\end{align*}
It is the same interpretation that satisfies the concept $C$ alone, which makes
sense because that interpretation satisfies the most left hand side part of
$\neg D$: $x : \neg B$.  Nevertheless, as we have found an interpretation that
satisfies $ C \sqcap \neg D $ then we have found and interpretation for which $
C^I \subseteq D^I $ is not true.  Therefore we can say that the subsumption $ C
\sqsubseteq D $ is \emph{not} true.

\subsection[Alternate]{Subsumption, alternate solution}
To check if the subsumption $ C \sqsubseteq D $ we can use another method.  As
to check this subsumption we need to check that in any interpretation $I$ the
concept $C$ is a subset of concept $D$ (i.e. $ C^I \subseteq D^I \mid \forall
I$) and we already know that $D$ is unsatisfiable, then we can say that $ D^I =
\emptyset \mid \forall I $.  In other words:
\begin{align*}
C   &\sqsubseteq D\\
C^I &\subseteq D^I        & \mid \forall I\\
C^I &\subseteq \emptyset  & \mid \forall I
\end{align*}

Now, if we find an $I$ for which $C^I$ is not the empty set we will see that $
C^I \subseteq \emptyset $ is not true $\forall I$.  As we already constructed
an interpretation $S$ for concept $C$ in section \ref{interpretationforc} we
can say that $ C^S \nsubseteq \emptyset $ and therefore we see:
\begin{align*}
C^I &\nsubseteq \emptyset  & \mid \forall I\\
C^I &\nsubseteq D^I        & \mid \forall I
\end{align*}

From this we can finally say that:
\begin{equation*}
C \not\sqsubseteq D
\end{equation*}

\section[Question 10]{Answer to question 10}

\begin{itemize}
\item[-]\verb|DisjointClasses(Meat Fish Vegetable Fruit)|
\begin{align*}
 Meat      \sqcap Fish      &\sqsubseteq \emptyset
&Fish      \sqcap Vegetable &\sqsubseteq \emptyset \\
 Meat      \sqcap Vegetable &\sqsubseteq \emptyset
&Fish      \sqcap Fruit     &\sqsubseteq \emptyset \\
 Meat      \sqcap Fruit     &\sqsubseteq \emptyset
&Vegetable \sqcap Fruit     &\sqsubseteq \emptyset
\end{align*}

\item[-]\verb|ObjectPropertyDomain(father Person)|
\begin{equation*}
\exists father.\top \sqsubseteq Person
\end{equation*}

\item[-]\verb|ObjectPropertyRange(father Male)|
\begin{equation*}
\exists father^-.\top \sqsubseteq Male
\end{equation*}

\item[-]\verb|SubObjectPropertyOf(father parent)|
\begin{equation*}
\exists father.\top \sqsubseteq \exists parent.\top
\end{equation*}

\item[-]\verb|FunctionalObjectProperty(father)|
\begin{equation*}
\exists father.\top \sqsubseteq {}\leq 1father.\top
\end{equation*}

\item[-]\verb|SymmetricObjectProperty(hasSameGrade)|
\begin{align*}
\exists hasSameGrade.\top   &\sqsubseteq \exists hasSameGrade^-.\top\\
\exists hasSameGrade^-.\top &\sqsubseteq \exists hasSameGrade.\top
\end{align*}

\item[-]\begin{Verbatim}[samepage=true]
SubClassOf(Lion ObjectIntersectionOf(
                    Animal
                    ObjectAllValuesFrom(eats Animal)))
\end{Verbatim}
\begin{equation*}
Lion \sqsubseteq Animal \sqcap \forall eats.Animal
\end{equation*}

\item[-]\begin{Verbatim}[samepage=true]
EquivalentClasses(Carnivore
                  ObjectIntersectionOf(
                      Animal
                      ObjectSomeValuesFrom(eats Animal)))
\end{Verbatim}
\begin{equation*}
Carnivore \equiv Animal \sqcap \exists eats.Animal
\end{equation*}
\end{itemize}

\end{document}

