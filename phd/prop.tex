\documentclass[a4paper,12pt]{article}

\usepackage[utf8]{inputenc}
\usepackage[T1]{fontenc}
\usepackage{textcomp}
\usepackage{parskip}

\title{Deep learning applied to multiomics: treating structured data as
unstructured}
\author{Michal Grochmal
  $<$\href{mailto:grochmal@member.fsf.org}{grochmal@member.fsf.org}$>$
}
\date{\today}

\usepackage[colorlinks=true]{hyperref}

\begin{document}
\maketitle
\thispagestyle{empty}

Proteomics and genomics data is very difficult to classify programaticaly.
Although multiomics databases posses huge amounts of data in structured schemas
we cannot be sure that the structure of these databases actually represent
features of the data.  The major example that the structure of the multiomics
databases do not represent all features in the proteomic or genomic data is the
fact that between distinct databases there are enormous differences in the
hierarchies used to structure the data.

We will try to prove the fact that the schemas of the multiomic databases
represent only a small part of the features that can be found in the multiomic
data.  If we can find a big amount of features in the multiomic data we can
evaluate the correlation and importance of these features to select the
features most relevant for the classification.  Using only the selected
features shall bestow the accuracy of the classification.  To perform this we
will use deep learning techniques over flattened data from multiomics
databases.

In the recent years deep learning came out of age, because of the availability
of powerful computers.  Deep learning shows promising results in automatically
learning features from unstructured data, notably in the computer vision field.
We will flatten the data on multiomics, consider it as if it were completely
unstructured data and feed it into the deep learning algorithm.  We shall find
low level and, more importantly, high level features that can be compared with
the hand crafted features of the data used to structure the database schemas.

Unfortunately I cannot build a sensible list of references.  This is a "mock"
proposal written in 4 hours, thanks to the fact that I was made aware of the
deadline for the proposal submission with only 24 hours of antecedence.
Apologies if I wasted your time by reading it.

\end{document}

