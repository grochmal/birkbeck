\documentclass[a4paper,12pt]{article}

\usepackage[utf8]{inputenc}
\usepackage[T1]{fontenc}
\usepackage{textcomp}

\usepackage{amsmath}
\usepackage{amssymb}
\usepackage{graphicx}

\title{2nd coursework for the Advances in Data Management module}
\author{Michal Grochmal
  $<$\href{mailto:grochmal@member.fsf.org}{grochmal@member.fsf.org}$>$
}
\date{\today}

\usepackage[pdftex,colorlinks=true]{hyperref}

\begin{document}
\maketitle

\section[Relational algebra]{Translate the query into relational algebra}
We can see the query as the following relational algebra expression:
\begin{equation*}
\pi_{name} (\sigma_{rid > 75}
             (Sailors \Join_{Sailors.sid = Reserves.sid} Reserves))
\end{equation*}

\section[Query plans]{Write down 4 query plans}
\begin{figure}[!htp]
\centering
  \includegraphics[width=0.48\textwidth]{queryA}
  \includegraphics[width=0.48\textwidth]{queryB}
\caption{Query plans A and B, these plans join the tables before selection.
\emph{Query plan A} first uses an \emph{index nested loop join (INLJ)} with the
outer table as \emph{Reserves} and the inner table \emph{Sailors}, it uses the
hash index on \emph{sid} on \emph{Sailors} as the index in the inner part of
INLJ.  Then the selection and projection are done \emph{on the fly}.
\emph{Query plan B} first sorts (in memory) the \emph{Sailors} table and stores
the sorted result in a temporary file.  Then uses a \emph{sort merge join
(SMJ)} to join the sorted \emph{Sailors} table with the \emph{Reserves} table,
the \emph{Reserves} table can be accessed sorted from the clustered B+tree
index on \emph{(sid, bid)} (this can be done because \emph{sid} is a prefix in
this index).  Finally selection and projection are done \emph{on the fly}.}
\label{figAB}
\end{figure}
\begin{figure}[!htp]
\centering
  \includegraphics[width=0.48\textwidth]{queryC}
  \includegraphics[width=0.48\textwidth]{queryD}
\caption{Query plans C and D, these plans perform selection before joining the
tables.  \emph{Query plan C} first selects all records from \emph{Reserves}
that have \emph{bid > 75} and feeds the result into the \emph{index nested loop
join} as the outer table.  In the INLJ \emph{Sailors} is again used as the
inner table with the index on \emph{sid}.  Following the join the projection is
done \emph{on the fly}.  \emph{Query plan D} also selects \emph{Reserves} on
\emph{bid > 75} first and then feeds it into the join.  But the join used is a
\emph{hash join (HJ)}.  After the join the projection in done \emph{on the
fly}.}
\label{figCD}
\end{figure}
Figures \ref{figAB} and \ref{figCD} shows four possible query plans for the
given query.

\section[Query cost]{Estimate the cost of each query plan}

We know that the number of pages of \emph{Sailors} is 5 and the number of pages
of \emph{Reserves} is 200.  We will use the symbols $N_S = 5$ and $N_R = 200$
to refer to these two numbers below.

\subsection[Query A]{Cost for query plan A}
Read the entire Reserves relation $ \Rightarrow 200 $ I/Os

For each of the 20 000 records in Reserves use the hash index on sid of the
Sailors relation to find corresponding records.  That hash index is clustered
therefore we need
$ \lceil selectivity(sid = Reserves.sid, Sailors) * N_S \rceil $ I/Os
for each record in Reserves.  The selectivity can be estimated to $ 1/50 $ so
we have a total of $ \lceil 1/50 * 5 \rceil $ I/Os per record in Reserves.
$ 20000 * \lceil 1/10 \rceil = 20000 $ I/Os.

Selection and projection are done on the fly as rows are generated from the
join.

Total cost: 20200 I/Os

\subsection[Query B]{Cost for query plan B}
Read the Sailors relation and sort it on sid (in memory) $ \Rightarrow 5 $ I/Os

Write the sorted records into a temporary file $ \Rightarrow 5 $ I/Os

Read Sailors from the temporary file and read Reserves from the clustered
B+tree index.  Both relations are sorted on sid therefore we can perform a sort
merge join directly (i.e. without sorting, we just merge).  The cost for the
merge will be approximately the cost to read the relations $ \Rightarrow 205 $.

Selection and projection are done on the fly.

Total cost: 215 I/Os

\subsection[Query C]{Cost for query plan C}
Read the Reserves relation and select it by bid > 75.  This requires a full
table scan as the B+tree index cannot be used (because bid is not a prefix in
this index).  $ \Rightarrow 200 $ I/Os

To estimate the number of records of Reserves that have bid > 75 we need the
selectivity of such records.  Given that the values of bid are uniformly
distributed we can say that:
$ selectivity \approx (100 - 75)/100 * 200 * 100 = 5000 $ records.

Now we perform INLJ using the resulting 5000 records from Reserves as the outer
loop and use the hash index on Sailors in the inner part.  This results in
$ 5000 * \lceil 1/50 * 5 \rceil = 5000 $ I/Os.

The projection is done on the fly.

Total cost: 5200 I/Os

\subsection[Query D]{Cost for query plan D}
Again we read the Reserves relation and select it by bid > 75.  As seen in the
previous section this results in approximately 5000 records.
$ \Rightarrow 200 $ I/Os.

There is no need to store the result in a temporary file as the next step (HJ)
can hash incoming record in any order.  We can pipeline directly into the HJ.

Hash join the selected 5000 records together with Sailors.  The 5000 records
can be stored over 50 pages (Reserves can be stored 100 record per page) and
Sailors have only 5 pages, therefore the cost of the HJ will be
$ 3 * (50 + 5) = 165 $ I/Os

The projection is done on the fly.

Total cost: 365 I/Os

\end{document}

